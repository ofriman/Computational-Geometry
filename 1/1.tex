\documentclass[a4paper, 8pt, oneside]{article}

 \usepackage[margin=1in]{geometry} 
\usepackage{amsmath,amsthm,amssymb, float,centernot}
\usepackage[shortlabels]{enumitem}
 \usepackage[hidelinks]{hyperref}
\usepackage{xcolor}
\hypersetup{
    colorlinks,
    linkcolor={red!50!black},
    citecolor={blue!50!black},
    urlcolor={blue!80!black}
}

\newtheorem{theorem}{Theorem}[section]
  
\newcommand{\N}{\mathbb{N}}
\newcommand{\Z}{\mathbb{Z}}
\newcommand{\R}{\mathbb{R}}
\newcommand\abs[1]{\left|#1\right|}
\newenvironment{sol}
    {\emph{Solution:}
    }
    {
    \qed
    }
\begin{document}

\title{Computational Geometry: Assignment 1}
\author{Oren Friman 301677613}
\maketitle

\medskip

\begin{enumerate}
\item
 Recall that for $p, q, r \in \R^2$ we defined $orient(p,q,r) = \text{sign det}
\begin{bmatrix}
    1 &1 & 1 \\
    p &q & r
\end{bmatrix}$.

Prove the following claims algebraically, without using any pictures:
 \begin{enumerate}
\item  \label{item:orient} $orient(p,q,r) = orient(q,r,p) = orient(r,p,q) = -orient(p,r,q) = -orient(q,p,r) = -orient(r,q,p)$

\begin{sol}
Recall that if $B$ is a matrix obtained from $A$ by interchanging two rows (columns), then $det B = - det A$
\begin{align*}
orient(p,q,r) &=  
\text{sign det}
\begin{bmatrix}
    1 &1 & 1 \\
    p &q & r
\end{bmatrix}\\&=
- \text{sign det}
\begin{bmatrix}
    1 &1 & 1 \\
   q &p & r
\end{bmatrix} && C_1 \leftrightarrow C_2 \\&=
\text{sign det}
\begin{bmatrix}
    1 &1 & 1 \\
   q &r & p
\end{bmatrix} =  orient(q,r,p) && C_2 \leftrightarrow C_3 \\&=
-\text{sign det}
\begin{bmatrix}
    1 &1 & 1 \\
   r &q & p
\end{bmatrix} && C_1 \leftrightarrow C_2 \\&=
\text{sign det}
\begin{bmatrix}
    1 &1 & 1 \\
   r &p & q
\end{bmatrix} =  orient(r,p,q) && C_2 \leftrightarrow C_3 \\&=
- \text{sign det}
\begin{bmatrix}
    1 &1 & 1 \\
   p &r & q
\end{bmatrix} =  - orient(p,r,q) && C_1 \leftrightarrow C_2 \\&=
\text{sign det}
\begin{bmatrix}
    1 &1 & 1 \\
   p &q & r
\end{bmatrix} = && C_2 \leftrightarrow C_3 \\&=
- \text{sign det}
\begin{bmatrix}
    1 &1 & 1 \\
   q &p & r
\end{bmatrix} =  - orient(q,p,r)  && C_1 \leftrightarrow C_2 \\&=
\text{sign det}
\begin{bmatrix}
    1 &1 & 1 \\
   q &r & p
\end{bmatrix}  = && C_2 \leftrightarrow C_3 \\&=
- \text{sign det}
\begin{bmatrix}
    1 &1 & 1 \\
   r &q & p
\end{bmatrix} =  - orient(r,q,p) && C_1 \leftrightarrow C_2
\end{align*}
\end{sol}

\item \label{itm:interiority} if $orient(p,x,q) =orient(q,x,r) = orient(r,x,p)$, then:
\begin{itemize}
\item $orient(r,q,p)$ is also equal to them;
\item $x$ is a convex combination of $p,q,r$.
\end{itemize}
\begin{sol}
Notice that
\begin{equation*} det
\begin{bmatrix}
    1 &1 & 1 & 1\\
    1 &1 & 1 & 1\\
    r &q & p & x
\end{bmatrix} = 0
\end{equation*}
because If $A$ is a matrix with two identical rows (or columns), then $det A = 0$.
Now lets expand the left hand side by the cofactors of its first row
\begin{align*}
det
\begin{bmatrix}
    1 &1 & 1 & 1\\
    1 &1 & 1 & 1\\
    r &q & p & x
\end{bmatrix} &=
det
\begin{bmatrix}
    1 & 1 & 1\\
    q& p & x
\end{bmatrix} -
det
\begin{bmatrix}
    1 & 1 & 1\\
    r & p& x
\end{bmatrix} +
det
\begin{bmatrix}
    1 & 1 & 1\\
    r & q & x
\end{bmatrix} -
det
\begin{bmatrix}
    1 & 1 & 1\\
    r & q & p
\end{bmatrix} \\&=
- \underbrace{det\begin{bmatrix}
    1 & 1 & 1\\
    p& q & x
\end{bmatrix}}_{C_1 \leftrightarrow C_2}
+ \underbrace{det
\begin{bmatrix}
    1 & 1 & 1\\
    r & x& p
\end{bmatrix}}_{C_2 \leftrightarrow C_3}
-\underbrace{det
\begin{bmatrix}
    1 & 1 & 1\\
    q & r & x
\end{bmatrix}}_{C_1 \leftrightarrow C_2} -
det
\begin{bmatrix}
    1 & 1 & 1\\
    r & q & p
\end{bmatrix}  \\&=
\underbrace{det\begin{bmatrix}
    1 & 1 & 1\\
    p& x & q
\end{bmatrix}}_{C_2 \leftrightarrow C_3}
+
\begin{bmatrix}
    1 & 1 & 1\\
    r & x& p
\end{bmatrix}
+ \underbrace{det
\begin{bmatrix}
    1 & 1 & 1\\
    q & x & r
\end{bmatrix}}_{C_2 \leftrightarrow C_3} -
det
\begin{bmatrix}
    1 & 1 & 1\\
    r & q & p
\end{bmatrix} = 0
\end{align*}
Adding $\small det
\begin{bmatrix}
    1 & 1 & 1\\
    r & q & p
\end{bmatrix}$ to  both side
\begin{equation}\label{interiority}
det
\begin{bmatrix}
    1 & 1 & 1\\
    r & q & p
\end{bmatrix} =
det
\begin{bmatrix}
    1 & 1 & 1\\
    p& x & q
\end{bmatrix} +
det
\begin{bmatrix}
    1 & 1 & 1\\
    r & x& p
\end{bmatrix} +
det
\begin{bmatrix}
    1 & 1 & 1\\
    q & x & r
\end{bmatrix} 
\end{equation}
Now assume $orient(p,x,q) =orient(q,x,r) = orient(r,x,p)$ and lets calculate $orient(r,q,p)$
\begin{align*}
orient(r,q,p) &= \text{sign det}
\begin{bmatrix}
    1 & 1 & 1\\
    r & q & p
\end{bmatrix} \\&= \text{sign} \Bigg(
det
\begin{bmatrix}
    1 & 1 & 1\\
    p& x & q
\end{bmatrix} +
det
\begin{bmatrix}
    1 & 1 & 1\\
    r & x& p
\end{bmatrix} +
det
\begin{bmatrix}
    1 & 1 & 1\\
    q & x & r
\end{bmatrix} 
\Bigg) && (\ref{interiority}) \\&=
 \text{sign det}
\begin{bmatrix}
    1 & 1 & 1\\
    p & x & q
\end{bmatrix} && \text{\small by assumption all signs are equal.}  \\&= orient(p,x,q).
\end{align*}
Now we can use Cramer's rule to calculate $x$ from combination of $p, q, r$
\begin{equation*}
\begin{bmatrix}
    1 & 1 & 1\\
    r &q & p
\end{bmatrix} t = \begin{bmatrix}
    1 \\ x
\end{bmatrix}
\end{equation*}
\begin{equation*}
\frac{det \begin{bmatrix}
    1 & 1 & 1\\
    x &q & p
\end{bmatrix}}{det \begin{bmatrix}
    1 & 1 & 1\\
    r &q & p
\end{bmatrix}} r +
\frac{det \begin{bmatrix}
    1 & 1 & 1\\
    r &x & p
\end{bmatrix}}{det \begin{bmatrix}
    1 & 1 & 1\\
    r &q & p
\end{bmatrix}} q +
\frac{det \begin{bmatrix}
    1 & 1 & 1\\
    r &q & x
\end{bmatrix}}{det \begin{bmatrix}
    1 & 1 & 1\\
    r &q & p
\end{bmatrix}} p = \begin{bmatrix}
    1 \\ x
\end{bmatrix}
\end{equation*}
Now let's calculate the sum of coefficients
\begin{align*}
\frac{det \begin{bmatrix}
    1 & 1 & 1\\
    x &q & p
\end{bmatrix}}{det \begin{bmatrix}
    1 & 1 & 1\\
    r &q & p
\end{bmatrix}} +
\frac{det \begin{bmatrix}
    1 & 1 & 1\\
    r &x & p
\end{bmatrix}}{det \begin{bmatrix}
    1 & 1 & 1\\
    r &q & p
\end{bmatrix}} +
\frac{det \begin{bmatrix}
    1 & 1 & 1\\
    r &q & x
\end{bmatrix}}{det \begin{bmatrix}
    1 & 1 & 1\\
    r &q & p
\end{bmatrix}} &=
\frac{det \begin{bmatrix}
    1 & 1 & 1\\
    x &q & p
\end{bmatrix} + det \begin{bmatrix}
    1 & 1 & 1\\
    r &x & p
\end{bmatrix} + det \begin{bmatrix}
    1 & 1 & 1\\
    r &q & x
\end{bmatrix}}{det \begin{bmatrix}
    1 & 1 & 1\\
    r &q & p
\end{bmatrix}} \\&= 
\frac{det \begin{bmatrix}
    1 & 1 & 1\\
    p &x & q
\end{bmatrix} + det \begin{bmatrix}
    1 & 1 & 1\\
    r &x & p
\end{bmatrix} + det \begin{bmatrix}
    1 & 1 & 1\\
    q &x & r
\end{bmatrix}}{det \begin{bmatrix}
    1 & 1 & 1\\
    r &q & p
\end{bmatrix}}
\\&= 1 && (\ref{interiority})
\end{align*}
Moreover recall that all coefficients have the same sign in the numerator and in the denominator so all of them most be positive, therefore this is convex combination.
\end{sol}
\item if $orient(p,x,q) = orient(p,x,r) = orient(p,x,s) = orient(q,x,r) = orient(r,x,s)$ then $orient(q,x,s)$ is also equal to them.

\begin{sol}
By contradiction assume that $orient(q,x,s)$ is not equal to them, by (\ref{item:orient}) we know that
\begin{equation}\label{item:orientchanges}
orient(q,x,s) = -orient(q,s,x) = orient(s,q,x) = -orient(s,x,q).
\end{equation}
From the assumtion and \ref{item:orientchanges} we get
\begin{equation*}
 orient(q,x,r) = orient(r,x,s) = orient(s,x,q)
\end{equation*}
So by (\ref{itm:interiority}) $x$ is a convex combination of $q,r,s$.
Now recall from the assumption that
\begin{equation*}
orient(p,x,q) = orient(p,x,r) = orient(p,x,s)
\end{equation*}
But this is impossible because $x$ is a convex combination of $q,r,s$.
\end{sol}
\end{enumerate}

\end{enumerate}
\end{document}